\documentclass[mathserif,serif]{beamer}
\usepackage[utf8]{inputenc}

\title{Weierstrass' Approximation Theorem and Bernstein Polynomials}
\author{Jishnu Kaiwar
  \and
  Bing En Gan
  \and
  Constantinos Azas}
\date{May 2019}
\subject{Mathematics}


\begin{document}
\frame{\titlepage}

\begin{frame}
  \frametitle{Approximating functions with polynomials.}
  \begin{itemize}
  \item<+-> Is it possible?
  \item<+-> Where is it possible?
  \item<+-> What do we mean by approximate?

  \end{itemize}
\end{frame}

\begin{frame}
  \frametitle{History of the Problem}
  \begin{itemize}
  \item 1885: Karl Weierstrass (1815-1897) establishes his Approximation Theorem, provides a general proof.
  \item 1908: Charles Poussin poses the question: \emph{Can we approximate functions with $n$-degree polynomials with an error less than $\frac{1}{n}$?}
  \item 1911: Sergei Bernstein solves this problem using \emph{Probability Theory}.
  \end{itemize}
\end{frame}


\begin{frame}{Weierstrass approximation theorem}
\begin{theorem}
The Weierstrass approximation theorem states that...
\end{theorem}
\end{frame}

\begin{frame}{Probabilistic Proof}
Consider an event $A$, where
$$\mathbb{P}(A)=x$$
Consider a continuous function $F$.
\newline 
Assume $n$ experiments are conducted and the player is paid the sum $F(\frac{m}{n})$ if the event $A$ occurs $m$ times.    
\end{frame}

\begin{frame}{Probabilistic Proof}
Now, calculate the expected value of the money received by the player.
\begin{equation*}
      E_n = \sum_{m=0}^{n} F \left( \frac{m}{n} \right) \binom{m}{n} x^n (1-x)^{n-m}
\end{equation*}
Note that this is a polynomial.
\end{frame}

\begin{frame}{Probabilistic Proof}
  By definition of continuity,
  \begin{equation*}
    (\exists\delta>0)(|x-x_0|<\delta)\implies(\forall\ebsilon>0)[]
  \end{equation*}
\end{frame}

\end{document}

%%% Local Variables:
%%% mode: latex
%%% TeX-master: t
%%% End:
