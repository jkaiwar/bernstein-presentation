\documentclass{beamer}
\usepackage[utf8]{inputenc}

\title{Weierstrass' Approximation Theorem and Bernstein Polynomials}
\author{Jishnu Kaiwar
  \and
  Bing En Gan
  \and
  Constantinos Azas}
\date{May 2019}
\subject{Mathematics}


\begin{document}
\frame{\titlepage}

\begin{frame}{Weierstrass approximation theorem}
\begin{theorem}
The Weierstrass approximation theorem states that...
\end{theorem}
\end{frame}

\begin{frame}{Probabilistic Proof}
Consider an event $A$, where
$$\mathbb{P}(A)=x$$
Consider a continuous function $F$.
\newline 
Assume $n$ experiments are conducted and the player is paid the sum $F(\frac{m}{n})$ if the event $A$ occurs $m$ times.    
\end{frame}

\begin{frame}{Probabilistic Proof}
Now, calculate the expected value of the money received by the player.
\begin{equation*}
      E_n = \sum_{m=0}^{n} F \left( \frac{m}{n} \right) \binom{m}{n} x^n (1-x)^{n-m}
\end{equation*}
Note that this is a polynomial.
\end{frame}

\begin{frame}{Probabilistic Proof}
By definition of continuity,
\begin{equation*}
    (\exists\delta>0)(|x-x_0|<\delta)\implies(\forall\ebsilon>0)[]
\end{equation*}
\end{frame}

\end{document}

%%% Local Variables:
%%% mode: latex
%%% TeX-master: t
%%% End:
